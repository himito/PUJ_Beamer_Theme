\documentclass[]{beamer}

% Tema de la Universidad
\usetheme{PUJ}

\usepackage[spanish]{babel}
\usepackage[utf8x]{inputenc}


\title{Título}
\subtitle{Subtítulo}
\author{Nombre del Autor}
\institute{Pontificia Universidad Javeriana}
\date{\today}

\begin{document}

% Frame del título
\begin{frame}[t,plain]
  	\titlepage
\end{frame}

\begin{frame}
	\frametitle{Contenido}
	\tableofcontents
\end{frame}

\section{Introducción}
\begin{frame}
  \frametitle{Introducción}
  
  Aquí va la \alert{introducción\footnote{pie de página}}
  
  \begin{enumerate}
	\item item 1
  	\item item 2
  \end{enumerate}
  
  \begin{itemize}
  	\item item1 
    \item item2 
  \end{itemize}

  \begin{description}
  	\item[descripcion1] bla bla bla
  	\item[descripcion2] bla bla bla 
  \end{description}
\end{frame}

\section{Frame1}
\begin{frame}
	\frametitle{Frame 1}
 
    \begin{exampleblock}{Ejemplo}
    	prueba bloque
  	\end{exampleblock}

  	\begin{block}{Bloque}
    	prueba bloque
  	\end{block}

  	\begin{alertblock}{Alerta}
    	prueba bloque
  	\end{alertblock}
\end{frame}


\section{Frame 2}
\begin{frame}
  \frametitle{Frame 2}
  \framesubtitle{logo de la Universidad}

  \begin{figure}[h]
    \centering
    \includegraphics[scale=0.5]{img/pujlogo.pdf}
    \caption{Logo Universidad}
  \end{figure}

\end{frame}


\section{Frame 3}
\begin{frame}
  \frametitle{Frame 3}
  \framesubtitle{tabla}

  \begin{table}[h]
    \centering
    \begin{tabular}{|c|c|}
      \hline
      \# Columna & \# columna \\
      \hline \hline
      columna 1 & columna2 \\
      columna 1 & columna2 \\
      \hline
    \end{tabular}
    \caption{Tabla prueba}
  \end{table}

\end{frame}


\end{document}